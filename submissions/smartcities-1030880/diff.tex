%  LaTeX support: latex@mdpi.com
%DIF LATEXDIFF DIFFERENCE FILE
%DIF DEL smartcities-1030880/template.tex   Mon Nov 23 21:11:34 2020
%DIF ADD ../docs/via_riders.tex             Sat Jan  9 13:28:01 2021
%  In case you need support, please attach all files that are necessary for compiling as well as the log file, and specify the details of your LaTeX setup (which operating system and LaTeX version / tools you are using).

%=================================================================
%DIF 5c5
%DIF < \documentclass[smartcities,article,submit,moreauthors,pdftex]{Definitions/mdpi} 
%DIF -------
\documentclass[smartcities,article,submit,moreauthors,pdftex]{mdpi} %DIF > 
%DIF -------

% If you would like to post an early version of this manuscript as a preprint, you may use preprint as the journal and change 'submit' to 'accept'. The document class line would be, e.g., \documentclass[preprints,article,accept,moreauthors,pdftex]{mdpi}. This is especially recommended for submission to arXiv, where line numbers should be removed before posting. For preprints.org, the editorial staff will make this change immediately prior to posting.

%DIF 9a9-18
%% Some pieces required from the pandoc template %DIF > 
\providecommand{\tightlist}{% %DIF > 
  \setlength{\itemsep}{0pt}\setlength{\parskip}{4pt}} %DIF > 
\setlist[itemize]{leftmargin=*,labelsep=5.8mm} %DIF > 
\setlist[enumerate]{leftmargin=*,labelsep=4.9mm} %DIF > 
 %DIF > 
\usepackage{longtable} %DIF > 
 %DIF > 
% see https://stackoverflow.com/a/47122900 %DIF > 
 %DIF > 
%DIF -------
%--------------------
% Class Options:
%--------------------
%----------
% journal
%----------
% Choose between the following MDPI journals:
%DIF 16c26
%DIF < % acoustics, actuators, addictions, admsci, aerospace, agriculture, agriengineering, agronomy, algorithms, animals, antibiotics, antibodies, antioxidants, applsci, arts, asc, asi, atmosphere, atoms, axioms, batteries, bdcc, behavsci , beverages, bioengineering, biology, biomedicines, biomimetics, biomolecules, biosensors, brainsci , buildings, cancers, carbon , catalysts, cells, ceramics, challenges, chemengineering, chemistry, chemosensors, children, cleantechnol, climate, clockssleep, cmd, coatings, colloids, computation, computers, condensedmatter, cosmetics, cryptography, crystals, dairy, data, dentistry, designs , diagnostics, diseases, diversity, drones, econometrics, economies, education, ejihpe, electrochem, electronics, energies, entropy, environments, epigenomes, est, fermentation, fibers, fire, fishes, fluids, foods, forecasting, forests, fractalfract, futureinternet, futurephys, galaxies, games, gastrointestdisord, gels, genealogy, genes, geohazards, geosciences, geriatrics, hazardousmatters, healthcare, heritage, highthroughput, horticulturae, humanities, hydrology, ijerph, ijfs, ijgi, ijms, ijns, ijtpp, informatics, information, infrastructures, inorganics, insects, instruments, inventions, iot, j, jcdd, jcm, jcp, jcs, jdb, jfb, jfmk, jimaging, jintelligence, jlpea, jmmp, jmse, jnt, jof, joitmc, jpm, jrfm, jsan, land, languages, laws, life, literature, logistics, lubricants, machines, magnetochemistry, make, marinedrugs, materials, mathematics, mca, medicina, medicines, medsci, membranes, metabolites, metals, microarrays, micromachines, microorganisms, minerals, modelling, molbank, molecules, mps, mti, nanomaterials, ncrna, neuroglia, nitrogen, notspecified, nutrients, ohbm, optics, particles, pathogens, pharmaceuticals, pharmaceutics, pharmacy, philosophies, photonics, physics, plants, plasma, polymers, polysaccharides, preprints , proceedings, processes, proteomes, psych, publications, quantumrep, quaternary, qubs, reactions, recycling, religions, remotesensing, reports, resources, risks, robotics, safety, sci, scipharm, sensors, separations, sexes, signals, sinusitis, smartcities, sna, societies, socsci, soilsystems, sports, standards, stats, surfaces, surgeries, sustainability, symmetry, systems, technologies, test, toxics, toxins, tropicalmed, universe, urbansci, vaccines, vehicles, vetsci, vibration, viruses, vision, water, wem, wevj
%DIF -------
% acoustics, actuators, addictions, admsci, aerospace, agriculture, agriengineering, agronomy, algorithms, animals, antibiotics, antibodies, antioxidants, applsci, arts, asc, asi, atmosphere, atoms, axioms, batteries, bdcc, behavsci , beverages, bioengineering, biology, biomedicines, biomimetics, biomolecules, biosensors, brainsci , buildings, cancers, carbon , catalysts, cells, ceramics, challenges, chemengineering, chemistry, chemosensors, children, cleantechnol, climate, clockssleep, cmd, coatings, colloids, computation, computers, condensedmatter, cosmetics, cryptography, crystals, dairy, data, dentistry, designs , diagnostics, diseases, diversity, drones, econometrics, economies, education, electrochem, electronics, energies, entropy, environments, epigenomes, est, fermentation, fibers, fire, fishes, fluids, foods, forecasting, forests, fractalfract, futureinternet, futurephys, galaxies, games, gastrointestdisord, gels, genealogy, genes, geohazards, geosciences, geriatrics, hazardousmatters, healthcare, heritage, highthroughput, horticulturae, humanities, hydrology, ijerph, ijfs, ijgi, ijms, ijns, ijtpp, informatics, information, infrastructures, inorganics, insects, instruments, inventions, iot, j, jcdd, jcm, jcp, jcs, jdb, jfb, jfmk, jimaging, jintelligence, jlpea, jmmp, jmse, jnt, jof, joitmc, jpm, jrfm, jsan, land, languages, laws, life, literature, logistics, lubricants, machines, magnetochemistry, make, marinedrugs, materials, mathematics, mca, medicina, medicines, medsci, membranes, metabolites, metals, microarrays, micromachines, microorganisms, minerals, modelling, molbank, molecules, mps, mti, nanomaterials, ncrna, neuroglia, nitrogen, notspecified, nutrients, ohbm, particles, pathogens, pharmaceuticals, pharmaceutics, pharmacy, philosophies, photonics, physics, plants, plasma, polymers, polysaccharides, preprints , proceedings, processes, proteomes, psych, publications, quantumrep, quaternary, qubs, reactions, recycling, religions, remotesensing, reports, resources, risks, robotics, safety, sci, scipharm, sensors, separations, sexes, signals, sinusitis, smartcities, sna, societies, socsci, soilsystems, sports, standards, stats, surfaces, surgeries, sustainability, symmetry, systems, technologies, test, toxics, toxins, tropicalmed, universe, urbansci, vaccines, vehicles, vetsci, vibration, viruses, vision, water, wem, wevj %DIF > 
%DIF -------

%---------
% article
%---------
% The default type of manuscript is "article", but can be replaced by:
% abstract, addendum, article, benchmark, book, bookreview, briefreport, casereport, changes, comment, commentary, communication, conceptpaper, conferenceproceedings, correction, conferencereport, expressionofconcern, extendedabstract, meetingreport, creative, datadescriptor, discussion, editorial, essay, erratum, hypothesis, interestingimages, letter, meetingreport, newbookreceived, obituary, opinion, projectreport, reply, retraction, review, perspective, protocol, shortnote, supfile, technicalnote, viewpoint
% supfile = supplementary materials

%----------
% submit
%----------
% The class option "submit" will be changed to "accept" by the Editorial Office when the paper is accepted. This will only make changes to the frontpage (e.g., the logo of the journal will get visible), the headings, and the copyright information. Also, line numbering will be removed. Journal info and pagination for accepted papers will also be assigned by the Editorial Office.

%------------------
% moreauthors
%------------------
% If there is only one author the class option oneauthor should be used. Otherwise use the class option moreauthors.

%---------
% pdftex
%---------
% The option pdftex is for use with pdfLaTeX. If eps figures are used, remove the option pdftex and use LaTeX and dvi2pdf.

%=================================================================
\firstpage{1}
\makeatletter
\setcounter{page}{\@firstpage}
\makeatother
\pubvolume{xx}
\issuenum{1}
\articlenumber{5}
%DIF 48-49c58-59
%DIF < \pubyear{2020}
%DIF < \copyrightyear{2020}
%DIF -------
\pubyear{2019} %DIF > 
\copyrightyear{2019} %DIF > 
%DIF -------
%\externaleditor{Academic Editor: name}
\history{Received: date; Accepted: date; Published: date}
%DIF 52c62
%DIF < %\updates{yes} % If there is an update available, un-comment this line
%DIF -------
\updates{yes} % If there is an update available, un-comment this line %DIF > 
%DIF -------

%% MDPI internal command: uncomment if new journal that already uses continuous page numbers
%\continuouspages{yes}

%------------------------------------------------------------------
% The following line should be uncommented if the LaTeX file is uploaded to arXiv.org
%\pdfoutput=1

%=================================================================
% Add packages and commands here. The following packages are loaded in our class file: fontenc, calc, indentfirst, fancyhdr, graphicx, lastpage, ifthen, lineno, float, amsmath, setspace, enumitem, mathpazo, booktabs, titlesec, etoolbox, amsthm, hyphenat, natbib, hyperref, footmisc, geometry, caption, url, mdframed, tabto, soul, multirow, microtype, tikz

%=================================================================
%DIF 65c75
%DIF < %% Please use the following mathematics environments: Theorem, Lemma, Corollary, Proposition, Characterization, Property, Problem, Example, ExamplesandDefinitions, Hypothesis, Remark, Definition, Notation, Assumption
%DIF -------
%% Please use the following mathematics environments: Theorem, Lemma, Corollary, Proposition, Characterization, Property, Problem, Example, ExamplesandDefinitions, Hypothesis, Remark, Definition %DIF > 
%DIF -------
%% For proofs, please use the proof environment (the amsthm package is loaded by the MDPI class).

%=================================================================
% Full title of the paper (Capitalized)
\Title{Rider Perceptions of an On-Demand Microtransit Service in Salt Lake County, Utah}

%DIF 72-75d82
%DIF < % Author Orchid ID: enter ID or remove command
%DIF < \newcommand{\orcidauthorA}{0000-0003-3999-7584} % Add \orcidA{} behind the author's name
%DIF < %\newcommand{\orcidauthorB}{0000-0000-000-000X} % Add \orcidB{} behind the author's name
%DIF < 
%DIF -------
% Authors, for the paper (add full first names)
%DIF 77c83
%DIF < \Author{Gregory S. Macfarlane$^{1*}$\orcidA{}, Christian Hunter$^{1}$, Austin Martinez$^{1}$, and Elizabeth Smith$^{1}$}
%DIF -------
\Author{Gregory. S Macfarlane$^{1, *}$\href{https://orcid.org/0000-0003-3999-7584}{\orcidicon}, Christian Hunter$^{1, \dagger}$, Austin Martinez$^{1}$, Elizabeth Smith$^{1, \ddagger}$} %DIF > 
%DIF -------

% Authors, for metadata in PDF
%DIF 80c86
%DIF < \AuthorNames{Gregory S. Macfarlane, Christian Hunter, Austin Martinez, and Elizabeth Smith}
%DIF -------
\AuthorNames{Gregory. S Macfarlane, Christian Hunter, Austin Martinez, Elizabeth Smith} %DIF > 
%DIF -------

% Affiliations / Addresses (Add [1] after \address if there is only one affiliation.)
%DIF 83-85c89-93
%DIF < \address[1]{%
%DIF < $^{1}$ \quad Department of Civil and Environmental Engineering, Brigham Young University}
%DIF < 
%DIF -------
\address{% %DIF > 
$^{1}$ \quad Brigham Young University %DIF > 
Civil and Environmental Engineering Department, %DIF > 
430 Engineering Building, Provo, Utah 84602; \href{mailto:gregmacfarlane@byu.edu}{\nolinkurl{gregmacfarlane@byu.edu}}\\ %DIF > 
} %DIF > 
%DIF -------
% Contact information of the corresponding author
%DIF 87c95
%DIF < \corres{Correspondence: gregmacfarlane@byu.edu; Tel.: +01-801-422-8505} 
%DIF -------
\corres{Correspondence: \href{mailto:gregmacfarlane@byu.edu}{\nolinkurl{gregmacfarlane@byu.edu}}; Tel.: +01-801-422-8505} %DIF > 
%DIF -------

% Current address and/or shared authorship
%DIF 90c98
%DIF < \firstnote{Current affiliation: University of Texas, Austin}
%DIF -------
\firstnote{Current affiliation: University of Texas at Austin} %DIF > 
%DIF -------
\secondnote{Current affiliation: Imperial College London}
%DIF 92d100
%DIF < % The commands \thirdnote{} till \eighthnote{} are available for further notes
%DIF -------

%DIF 94d101
%DIF < %\simplesumm{} % Simple summary
%DIF -------

%DIF 96d102
%DIF < %\conference{} % An extended version of a conference paper
%DIF -------

%DIF 98a103-109
 %DIF > 
 %DIF > 
 %DIF > 
% The commands \thirdnote{} till \eighthnote{} are available for further notes %DIF > 
 %DIF > 
% Simple summary %DIF > 
 %DIF > 
%DIF -------
% Abstract (Do not insert blank lines, i.e. \\)
%DIF 99c111-127
%DIF < \abstract{On-demand microtransit services are frequently seen as an important tool in supporting first and last mile operations surrounding fixed route high frequency transit facilities, but questions remain surrounding who will use these novel services and for what purposes. In November 2019, the Utah Transit Authority launched an on-demand microtransit service in south Salt Lake County in partnership with a private mobility operator. This paper reports the results of a survey of 130 transit riders in the microtransit service area collected before and immediately after the service launched. There is not a clear relationship between current transit access mode and expressed willingness to use microtransit, though some responses from new riders indicate the novel service competes most directly with commercial transportation network company operations. The survey responses also reveal younger passengers express more than expected willingness to use microtransit, middle-aged passengers a less than expected willingness, and older passengers neutral or no expressed opinion. The effect of other user characteristics including income and automobile availability is less statistically clear and requires further research.}
%DIF -------
\abstract{On-demand microtransit services are frequently seen as an important tool in %DIF > 
supporting first and last mile operations surrounding fixed route high %DIF > 
frequency transit facilities, but questions remain surrounding who will use %DIF > 
these novel services and for what purposes. In November 2019, the Utah Transit %DIF > 
Authority launched an on-demand microtransit service in south Salt Lake County %DIF > 
in partnership with a private mobility operator. This paper reports the %DIF > 
results of a survey of 130 transit riders in the microtransit service area %DIF > 
collected before and immediately after the service launched. There is not a %DIF > 
clear relationship between current transit access mode and expressed %DIF > 
willingness to use microtransit, though some responses from new riders %DIF > 
indicate the novel service competes most directly with commercial %DIF > 
transportation network company operations. The survey responses also reveal %DIF > 
younger passengers express more than expected willingness to use microtransit, %DIF > 
middle-aged passengers a less than expected willingness, and older passengers %DIF > 
neutral or no expressed opinion. The effect of other user characteristics %DIF > 
including income and automobile availability is less statistically clear and %DIF > 
requires further research.} %DIF > 
%DIF -------

% Keywords
\keyword{on-demand transit; microtransit}

% The fields PACS, MSC, and JEL may be left empty or commented out if not applicable
%\PACS{J0101}
%\MSC{}
%\JEL{}

%%%%%%%%%%%%%%%%%%%%%%%%%%%%%%%%%%%%%%%%%%
% Only for the journal Diversity
%\LSID{\url{http://}}

%%%%%%%%%%%%%%%%%%%%%%%%%%%%%%%%%%%%%%%%%%
% Only for the journal Applied Sciences:
%\featuredapplication{Authors are encouraged to provide a concise description of the specific application or a potential application of the work. This section is not mandatory.}
%%%%%%%%%%%%%%%%%%%%%%%%%%%%%%%%%%%%%%%%%%

%%%%%%%%%%%%%%%%%%%%%%%%%%%%%%%%%%%%%%%%%%
% Only for the journal Data:
%\dataset{DOI number or link to the deposited data set in cases where the data set is published or set to be published separately. If the data set is submitted and will be published as a supplement to this paper in the journal Data, this field will be filled by the editors of the journal. In this case, please make sure to submit the data set as a supplement when entering your manuscript into our manuscript editorial system.}

%\datasetlicense{license under which the data set is made available (CC0, CC-BY, CC-BY-SA, CC-BY-NC, etc.)}

%%%%%%%%%%%%%%%%%%%%%%%%%%%%%%%%%%%%%%%%%%
% Only for the journal Toxins
%\keycontribution{The breakthroughs or highlights of the manuscript. Authors can write one or two sentences to describe the most important part of the paper.}

%\setcounter{secnumdepth}{4}
%%%%%%%%%%%%%%%%%%%%%%%%%%%%%%%%%%%%%%%%%%
 %DIF > 
 %DIF > 
\usepackage{booktabs} %DIF > 
%DIF PREAMBLE EXTENSION ADDED BY LATEXDIFF
%DIF UNDERLINE PREAMBLE %DIF PREAMBLE
\RequirePackage[normalem]{ulem} %DIF PREAMBLE
\RequirePackage{color}\definecolor{RED}{rgb}{1,0,0}\definecolor{BLUE}{rgb}{0,0,1} %DIF PREAMBLE
\providecommand{\DIFadd}[1]{{\protect\color{blue}\uwave{#1}}} %DIF PREAMBLE
\providecommand{\DIFdel}[1]{{\protect\color{red}\sout{#1}}}                      %DIF PREAMBLE
%DIF SAFE PREAMBLE %DIF PREAMBLE
\providecommand{\DIFaddbegin}{} %DIF PREAMBLE
\providecommand{\DIFaddend}{} %DIF PREAMBLE
\providecommand{\DIFdelbegin}{} %DIF PREAMBLE
\providecommand{\DIFdelend}{} %DIF PREAMBLE
\providecommand{\DIFmodbegin}{} %DIF PREAMBLE
\providecommand{\DIFmodend}{} %DIF PREAMBLE
%DIF FLOATSAFE PREAMBLE %DIF PREAMBLE
\providecommand{\DIFaddFL}[1]{\DIFadd{#1}} %DIF PREAMBLE
\providecommand{\DIFdelFL}[1]{\DIFdel{#1}} %DIF PREAMBLE
\providecommand{\DIFaddbeginFL}{} %DIF PREAMBLE
\providecommand{\DIFaddendFL}{} %DIF PREAMBLE
\providecommand{\DIFdelbeginFL}{} %DIF PREAMBLE
\providecommand{\DIFdelendFL}{} %DIF PREAMBLE
%DIF LISTINGS PREAMBLE %DIF PREAMBLE
\RequirePackage{listings} %DIF PREAMBLE
\RequirePackage{color} %DIF PREAMBLE
\lstdefinelanguage{DIFcode}{ %DIF PREAMBLE
%DIF DIFCODE_UNDERLINE %DIF PREAMBLE
  moredelim=[il][\color{red}\sout]{\%DIF\ <\ }, %DIF PREAMBLE
  moredelim=[il][\color{blue}\uwave]{\%DIF\ >\ } %DIF PREAMBLE
} %DIF PREAMBLE
\lstdefinestyle{DIFverbatimstyle}{ %DIF PREAMBLE
	language=DIFcode, %DIF PREAMBLE
	basicstyle=\ttfamily, %DIF PREAMBLE
	columns=fullflexible, %DIF PREAMBLE
	keepspaces=true %DIF PREAMBLE
} %DIF PREAMBLE
\lstnewenvironment{DIFverbatim}{\lstset{style=DIFverbatimstyle}}{} %DIF PREAMBLE
\lstnewenvironment{DIFverbatim*}{\lstset{style=DIFverbatimstyle,showspaces=true}}{} %DIF PREAMBLE
%DIF END PREAMBLE EXTENSION ADDED BY LATEXDIFF

\begin{document}
%%%%%%%%%%%%%%%%%%%%%%%%%%%%%%%%%%%%%%%%%%

%DIF < %%%%%%%%%%%%%%%%%%%%%%%%%%%%%%%%%%%%%%%%%
\DIFaddbegin \hypertarget{intro}{%
\section{Introduction}\label{intro}}
\DIFaddend 

\DIFdelbegin \section{\DIFdel{Introduction}}
%DIFAUXCMD
\addtocounter{section}{-1}%DIFAUXCMD
\DIFdelend Transit ridership in the United States has been in decline over the last several
years, with underlying causes ranging from service cuts to the advent of new
mobility options \citep{Graehler2019, Mallett2018}. These new mobility options \DIFdelbegin \DIFdel{– }\DIFdelend \DIFaddbegin \DIFadd{--
}\DIFaddend including bikeshare, e-scooters, and ridehailing through Transportation Network
Companies (TNCs) \DIFdelbegin \DIFdel{– }\DIFdelend \DIFaddbegin \DIFadd{-- }\DIFaddend might also play an important role in supporting transit
operations if the relative strengths of transit and modern mobility systems can
be successfully partnered \DIFaddbegin \DIFadd{\mbox{%DIFAUXCMD
\citep{Shaheen2016, OOSTENDORP201872, ShivThesis}}\hspace{0pt}%DIFAUXCMD
}\DIFaddend .

One particular area where a partnership between high-capacity, fixed-route
transit and TNC operations has been desired is in supporting first mile / last
mile operations in low-density suburban regions \citep{Shaheen2016, alonso2018, Kang2020}. TNC operators are incentivized to operate in dense areas where many
potential passengers are located \citep{Wong2020}, meaning they compete with transit
where transit can be most successful. But regulations or partnerships that
changed this incentive pattern could be highly beneficial to many transit riders
\citep{Ronald2017, Deakin2010}. For example, a transit agency might partner with a
TNC to offer shared rides at a subsidized fare in low-density areas where fixed
route transit services are ineffective or expensive. As these partnerships to
offer microtransit services materialize through demonstration projects or
permanent offerings, there is an important opportunity to observe and evaluate
who is using the service and for what reasons. It is also valuable to understand
how users perceive the effectiveness and convenience of these systems.

In this paper we report the results of a survey conducted immediately before and
several weeks after the November 2019 launch of a microtransit service in south
Salt Lake County, Utah by the Utah Transit Authority (UTA). The surveys were
designed to understand first the awareness of the on-demand system in the
transit passenger community. The surveys also consider the stated and revealed
likeliness of individuals to use the microtransit service, and how the
characteristics of these individuals \DIFdelbegin \DIFdel{– }\DIFdelend \DIFaddbegin \DIFadd{-- }\DIFaddend particularly age and household size \DIFdelbegin \DIFdel{– }\DIFdelend \DIFaddbegin \DIFadd{--
}\DIFaddend influence these preferences.

The remainder of this section contains a brief review of previous and ongoing
studies relevant to the question of demand for and use of microtransit services.
We then describe the survey methodology for this study, including both the
context of the UTA microtransit service as well as the survey instrument and
collection strategy. The survey results in several dimensions are followed by a
discussion of the limitations of the findings and associated opportunities for
future research.

%DIF < %%%%%%%%%%%%%%%%%%%%%%%%%%%%%%%%%%%%%%%%%
\DIFdelbegin \subsection{\DIFdel{Findings from Other Systems}}
%DIFAUXCMD
\addtocounter{subsection}{-1}%DIFAUXCMD
\DIFdelend \DIFaddbegin \hypertarget{findings-from-other-systems}{%
\subsection{Findings from Other Systems}\label{findings-from-other-systems}}

\DIFaddend In the last few years a number of on-demand microtransit services have begun
operations in many cities around the world. Given the dynamic nature of this
space, the literature is not mature and numerous projects are under evaluation
at the moment. However, some findings from early systems are available and are
worthy of discussion.

A microtransit service in Helsinki, Finland known as \DIFdelbegin \DIFdel{“Kutsuplus” }\DIFdelend \DIFaddbegin \DIFadd{``Kutsuplus'' }\DIFaddend operated from
2012 to 2015 and has been the subject of a number of studies. \citet{weckstrom2018} and
\citet{Haglund2019} each conduct a comprehensive analysis of the system using rider
questionnaires supplemented with GPS data points. The studies found that the
system was used by a wide variety of individuals for a wide variety of trip
purposes, and the typical trip length suggested it was being used less like a
taxi service and more to supplement last-mile transit access. In many cases, it
appeared as though Kutsuplus replaced walking and bicycle trips. The
\DIFdelbegin \DIFdel{\mbox{%DIFAUXCMD
\citeauthor{weckstrom2018} }\hspace{0pt}%DIFAUXCMD
}\DIFdelend \DIFaddbegin \DIFadd{\mbox{%DIFAUXCMD
\citet{weckstrom2018} }\hspace{0pt}%DIFAUXCMD
}\DIFaddend research also asked respondents why they continued or
discontinued using the service, revealing strong differences in response among
different income groups. High-income individuals were more likely to cite long
response times, while lower income groups were more likely to cite the fare or
difficulties understanding the service, or even not being aware of its
existence.

\citet{alonso2018} examined a microtransit system in the Arnhem-Nijmegen region in the
Netherlands. They develop a methodology to calculate the accessibility
contributed by the microtransit system above and beyond that provided by the
fixed route transit system, and their findings suggest the microtransit
service substantively enhances the mobility of people in the region. In this
study the authors use GPS trip data from the service and do not have access to
the actual riders to understand their preferences or characteristics.

In 2016 Austin, Texas, introduced a TNC operated as a non-profit and called
\DIFdelbegin \DIFdel{“RideAustin.” }\DIFdelend \DIFaddbegin \DIFadd{``RideAustin.'' }\DIFaddend The unique corporate structure of this TNC encourages it to share
data from the system with researchers, leading to a number of studies examining
the trip patterns of its users. \citet{Komanduri2018} show that a high proportion of
trips (60\%) taken on RideAustin could have been completed with a single-seat
transit ride. \citet{Wenzel2019} additionally used the same dataset to estimate the
level of deadheading and concomitant energy expenditure on the system. Though
these findings are important in terms of understanding the risks of microtransit
services, it should be stressed that the RideAustin was not explicitly designed
to support transit operations. And although the RideAustin dataset does identify
unique individual riders through a persistent mobile device ID, it does not
disclose any demographic information on the riders and therefore cannot support
an analysis of their characteristics or preferences.

\DIFaddbegin \DIFadd{\mbox{%DIFAUXCMD
\citet{KONIG2020954} }\hspace{0pt}%DIFAUXCMD
present a survey focused on determining preferences and
attitudes towards demand-responsive transit use in two rural regions in
Germany. A structural equations model of expressed preferences suggests that
users' attitudes are most powerfully driven by the expected performance of the
system in terms of wait and travel time, and less materially by attitudes
towards other public transit systems or social perspectives. This is valuable
insight, but attitudes such as these are difficult to forecast for a population,
and therefore difficult to incorporate into service planning exercises. The
authors collected demographic characteristics of the survey respondents, but did
not consider these characteristics in the statistical models.
}

\DIFaddend The literature to this point has been greatly aided by the use of so-called Big
Data: GPS records, rider transaction data, and the like. These data are
well-suited to important research questions such as where and when the services
pick up and drop off riders, the wait times experienced by the riders, and in
some cases even the ability to construct multiple trip tours. But the literature
to this point is somewhat limited in its exploration of the actual users of
these systems: who they are, why they are traveling, and why they chose to use
this service. \DIFaddbegin \DIFadd{This information is critical when planning and forecasting the
potential success or failure of these systems, in contrast to reporting observed
service characteristics in a service already in operation. In this paper, we
present the results of a rider survey designed to answer these questions in the
periods immediately before and after the launch of this kind of system.
}\DIFaddend 

%DIF < %%%%%%%%%%%%%%%%%%%%%%%%%%%%%%%%%%%%%%%%%
\DIFdelbegin \section{\DIFdel{Study Methodology}}
%DIFAUXCMD
\addtocounter{section}{-1}%DIFAUXCMD
\subsection{\DIFdel{System Description}}
%DIFAUXCMD
\addtocounter{subsection}{-1}%DIFAUXCMD
\DIFdelend \DIFaddbegin \hypertarget{study-methodology}{%
\section{Study Methodology}\label{study-methodology}}

\hypertarget{system-description}{%
\subsection{System Description}\label{system-description}}

\DIFaddend In November 2019, the Utah Transit Authority (UTA) launched an on-demand
microtransit service in the southern part of Salt Lake County. This region \DIFdelbegin \DIFdel{– }\DIFdelend \DIFaddbegin \DIFadd{--
}\DIFaddend illustrated in Figure \ref{fig:via-map} \DIFdelbegin \DIFdel{– }\DIFdelend \DIFaddbegin \DIFadd{-- }\DIFaddend has primarily low-density suburban
development but also hosts stations for UTA's extensive rail transit network:
the FrontRunner commuter rail operates between Provo and Ogden via downtown Salt
Lake City on 30 minute peak headways; and the Blue and Red TRAX light rail lines
connect to downtown Salt Lake City, the University of Utah, and Salt Lake
International Airport (via transfer) on 15 minute peak headways. There are
existing fixed route and route deviation services in the region, as well as park
and ride facilities at most rail stations. UTA is interested in improving the
quality of service for passengers in the region as well as reducing
per-passenger operating costs.

\begin{figure}
\DIFdelbeginFL %DIFDELCMD < \centering
%DIFDELCMD <     \includegraphics[width = \textwidth]{service_area.pdf}
%DIFDELCMD <     %%%
\DIFdelendFL \DIFaddbeginFL \includegraphics[width=1\linewidth]{images/service_area} \DIFaddendFL \caption{UTA on-demand microtransit service area. Image by the authors\DIFdelbeginFL \DIFdelFL{, using data from Utah AGRC and OpenStreetMap.}\DIFdelendFL }\label{fig:via-map}
\end{figure}

In establishing the on-demand microtransit service UTA partnered with Via, a
commercial mobility provider with new and ongoing operations in several US
cities. Passengers request rides using the VIA mobile application or calling a
designated service line and await the vehicle at a pickup point near to their
origin. Passengers share rides based on the availability of vehicles and the
compatibility of paths, as determined by algorithms embedded in the VIA service.
The vehicle will drop the passenger off near their destination or at TRAX or
FrontRunner stations; both the pickup and drop-off points must lie within the
service area shown in Figure \ref{fig:via-map}. The regular adult one-way fare
is \$2.50 and includes a limited transfer to the UTA fixed route transit system.
By the end of February 2020, the microtransit system was carrying about 316
passenger trips per weekday with an average wait time of 11 minutes per trip
\DIFdelbegin \DIFdel{\mbox{%DIFAUXCMD
\citep{uta2020}}\hspace{0pt}%DIFAUXCMD
}\DIFdelend \DIFaddbegin \DIFadd{\mbox{%DIFAUXCMD
\citet{uta2020}}\hspace{0pt}%DIFAUXCMD
}\DIFaddend .

\DIFdelbegin \subsection{\DIFdel{Survey Design}}
%DIFAUXCMD
\addtocounter{subsection}{-1}%DIFAUXCMD
\DIFdel{UTA’}\DIFdelend \DIFaddbegin \hypertarget{survey-design}{%
\subsection{Survey Design}\label{survey-design}}

\DIFadd{UTA'}\DIFaddend s primary goal in executing this survey was to understand the effectiveness
of its marketing campaign to raise awareness and information of the new service.
This survey also provided an opportunity to inform additional riders and to
evaluate rider perceptions and characteristics both before and immediately after
the service launch. As such the survey was administered in two tranches. The
first tranche was conducted on November 6th, 13th, and 14th of 2019 through
on-platform intercept interviews at the Draper and South Jordan FrontRunner
stations as well as the Draper Town Center TRAX station. The second tranche was
collected on several weekdays between January 10th and March 4th, 2020, and was
collected through on-platform intercept interviews at the same stations in
addition to the Daybreak Parkway TRAX station and at designated microtransit
pick-up points near the aforementioned rail stations; a limited number of
interviews were also conducted on board the microtransit vehicles. Interviews
were conducted throughout the day, but with a focus on the PM peak commute
period\DIFdelbegin \DIFdel{. The number of interviews conducted during each time period for each tranche is shown in Table \ref{tab:survey-times}.
}%DIFDELCMD < 

%DIFDELCMD < \begin{table}[ht]
%DIFDELCMD <     \centering
%DIFDELCMD <     %%%
%DIFDELCMD < \caption{%
{%DIFAUXCMD
\DIFdelFL{Surveys Collected by Time of Day}}
    %DIFAUXCMD
%DIFDELCMD < \label{tab:survey-times}
%DIFDELCMD <     \begin{tabular}{lcc}
%DIFDELCMD <     \toprule
%DIFDELCMD <     %%%
\DIFdelFL{Day Period           }%DIFDELCMD < & %%%
\DIFdelFL{Before   Launch }%DIFDELCMD < & %%%
\DIFdelFL{After   Launch }%DIFDELCMD < \\
%DIFDELCMD <     \midrule
%DIFDELCMD <     %%%
\DIFdelFL{AM (6-10)            }%DIFDELCMD < & %%%
\DIFdelendFL \DIFaddbeginFL \DIFaddFL{; approximately 60\% of the surveys in both tranches were collected between 4 and }\DIFaddendFL 7 \DIFdelbeginFL %DIFDELCMD < & %%%
\DIFdelFL{6              }%DIFDELCMD < \\
%DIFDELCMD <     %%%
\DIFdelFL{Mid-Day (10-4)       }%DIFDELCMD < & %%%
\DIFdelFL{13              }%DIFDELCMD < & %%%
\DIFdelFL{26             }%DIFDELCMD < \\
%DIFDELCMD <     %%%
\DIFdelFL{PM(4-7)             }%DIFDELCMD < & %%%
\DIFdelFL{33              }%DIFDELCMD < & %%%
\DIFdelFL{43             }%DIFDELCMD < \\
%DIFDELCMD <     %%%
\DIFdelFL{Evening (7-Midnight) }%DIFDELCMD < & %%%
\DIFdelFL{2               }%DIFDELCMD < & %%%
\DIFdelFL{0              }%DIFDELCMD < \\
%DIFDELCMD <     %%%
\DIFdelFL{TOTAL                }%DIFDELCMD < & %%%
\DIFdelFL{55              }%DIFDELCMD < & %%%
\DIFdelFL{75            }%DIFDELCMD < \\
%DIFDELCMD <     \bottomrule
%DIFDELCMD <     \end{tabular}
%DIFDELCMD < \end{table}
%DIFDELCMD < %%%
\DIFdelend \DIFaddbegin \DIFadd{PM.
}\DIFaddend 

The surveys were administered via electronic tablet using a questionnaire
developed in a web-based survey software. The survey questions were developed
with the help of UTA staff and an external consulting team. The relevant
variables and source questions for this study are shown in Table
\DIFdelbegin \DIFdel{\ref{tab:survey-summary}}\DIFdelend \DIFaddbegin \DIFadd{\ref{tab:survey-questions}}\DIFaddend , in the order in which the questions were asked. After
asking the respondent about their awareness of the system, the interviewer would
give a brief explanation of the service before asking about the respondent\DIFdelbegin \DIFdel{’}\DIFdelend \DIFaddbegin \DIFadd{'}\DIFaddend s
likeliness to use the system. The questionnaire for the second tranche included
additional questions that were identified as being important after the first
tranche was collected; for example, the questions about income and household
size were added between the tranches. Further, questions in the second tranche
for respondents on train platforms and either at or on board the microtransit
service had slightly different wording to reflect the separate contexts. There
was also a set of questions requesting general feedback on the UTA service that
is not included in this study.

\DIFdelbegin %DIFDELCMD < \begin{table}[ht]
%DIFDELCMD < \renewcommand{\arraystretch}{1.5}
%DIFDELCMD <     %%%
\DIFdelendFL \DIFaddbeginFL \begin{table}

\caption{\label{tab:survey-questions}\DIFaddFL{Survey Questionnaire Summary}}
\DIFaddendFL \centering
\DIFdelbeginFL %DIFDELCMD < \caption{%
{%DIFAUXCMD
\DIFdelFL{Survey Questionnaire Summary}}
    %DIFAUXCMD
%DIFDELCMD < \label{tab:survey-summary}
%DIFDELCMD < \begin{tabular}{l p{0.4\textwidth}p{0.3\textwidth}}
%DIFDELCMD < %%%
\DIFdelendFL \DIFaddbeginFL \begin{tabular}[t]{lp{0.4\textwidth}p{0.4\textwidth}}
\DIFaddendFL \toprule
Variable & Question Text & Response Type\\
\midrule
Frequency & How often do you ride UTA? & Multiple choice with days per week\\
Purpose & Where are you headed today? & Multiple choice with purposes plus text "other"\\
Access Mode & How did you travel to your UTA stop/station today? & Multiple choice with modes plus text "other"\\
Awareness & Had you heard about UTA On Demand before today? & Yes / No\\
Likeliness & How likely are you to download the VIA app and use UTA On Demand? & Likert scale with five "likely" levels\\
\DIFaddbeginFL \addlinespace
\DIFaddendFL Why Likely & Why did you choose that ranking? & Text response\\
Use Purpose & What types of trips do you think you could use it for? & Multiple choice with purposes plus text "other"\\
Auto Availability & How many vehicles (cars, trucks or motorcycles) are available in your household? & Multiple choice with 0 through 4+\\
Household Size & Including you, how many people live in your household? & \DIFdelbeginFL \DIFdelFL{Numeric                                           }\DIFdelendFL \DIFaddbeginFL \DIFaddFL{Multiple choice with 0 through 4+}\DIFaddendFL \\
Race & What is your race / ethnicity? & Mutiple choice allowing multiple selection\\
\DIFaddbeginFL \addlinespace
\DIFaddendFL Income & Which of the following BEST describes your TOTAL ANNUAL HOUSEHOLD INCOME in 2019 before taxes? & Multiple choice in ranges\\
Smartphone & Do you have a smartphone? & Yes / No\\
Age & What is your age? & Multiple choice in ranges\\
\bottomrule
\end{tabular}
\end{table}

%DIF < %%%%%%%%%%%%%%%%%%%%%%%%%%%%%%%%%%%%%%%%%
\DIFdelbegin \section{\DIFdel{Results}}
%DIFAUXCMD
\addtocounter{section}{-1}%DIFAUXCMD
\DIFdelend \DIFaddbegin \hypertarget{results}{%
\section{Results}\label{results}}
\DIFaddend 

The surveyors conducted 55 interviews in the first tranche and 75 in the second
tranche; the second tranche consisted of 58 interviews on rail transit platforms
and 17 interviews on the mictrotransit vehicles or at the microtransit pick-up
point adjacent to the rail stations. A summary of the survey respondents in each
tranche is given in Table \ref{tab:survey-respondents}; as outlined in the
Methodology section, the decision to include income level in the survey was made
between the tranches and therefore the \DIFdelbegin \DIFdel{“Before” }\DIFdelend \DIFaddbegin \DIFadd{``Before'' }\DIFaddend tranche contains no income
information. The number of respondents who declined to answer the other
demographic questions is also relatively high.

\DIFdelbegin %DIFDELCMD < \begin{table}[ht]
%DIFDELCMD <     \centering
%DIFDELCMD <     %%%
%DIFDELCMD < \caption{%
{%DIFAUXCMD
\DIFdelFL{Demographic Characteristics of Survey Respondents}}
    %DIFAUXCMD
%DIFDELCMD < \label{tab:survey-respondents}
%DIFDELCMD <  \renewcommand{\arraystretch}{1.5}
%DIFDELCMD <    %%%
\DIFdelendFL \DIFaddbeginFL \begin{table}
\DIFaddendFL 

\DIFdelbeginFL %DIFDELCMD < \begin{tabular}{@{}lcc@{}}
%DIFDELCMD < %%%
\DIFdelendFL \DIFaddbeginFL \caption{\label{tab:survey-respondents}\DIFaddFL{Demographic Characteristics of Survey Respondents}}
\centering
\begin{tabular}[t]{llllll}
\DIFaddendFL \toprule
\DIFaddbeginFL \multicolumn{2}{c}{ } \DIFaddendFL & \DIFdelbeginFL \DIFdelFL{Before (n   = 55)}\DIFdelendFL \DIFaddbeginFL \multicolumn{2}{c}{Before (N=55)} \DIFaddendFL & \DIFdelbeginFL \DIFdelFL{After (n   = 75)}\DIFdelendFL \DIFaddbeginFL \multicolumn{2}{c}{After (N=75)} \DIFaddendFL \\
\DIFaddbeginFL \cmidrule\DIFaddFL{(l}{\DIFaddFL{3pt}}\DIFaddFL{r}{\DIFaddFL{3pt}}\DIFaddFL{)}{\DIFaddFL{3-4}} \cmidrule\DIFaddFL{(l}{\DIFaddFL{3pt}}\DIFaddFL{r}{\DIFaddFL{3pt}}\DIFaddFL{)}{\DIFaddFL{5-6}}
  &    & \DIFaddFL{N }& \DIFaddFL{\% }& \DIFaddFL{N  }& \DIFaddFL{\% }\\
\DIFaddendFL \midrule
\DIFdelbeginFL \emph{\DIFdelFL{Smartphone}}         %DIFAUXCMD
\DIFdelendFL \DIFaddbeginFL \DIFaddFL{Smartphone }\DIFaddendFL & \DIFaddbeginFL \DIFaddFL{No }\DIFaddendFL & \DIFaddbeginFL \DIFaddFL{3 }& \DIFaddFL{2.3 }& \DIFaddFL{2 }& \DIFaddFL{1.5}\DIFaddendFL \\
 \DIFdelbeginFL %DIFDELCMD < \quad %%%
\DIFdelendFL \DIFaddbeginFL & \DIFaddendFL Yes & 42 & \DIFaddbeginFL \DIFaddFL{32.3 }& \DIFaddendFL 48 \DIFaddbeginFL & \DIFaddFL{36.9}\DIFaddendFL \\
 \DIFdelbeginFL %DIFDELCMD < \quad %%%
\DIFdelFL{No                 }\DIFdelendFL & \DIFdelbeginFL \DIFdelFL{3                 }\DIFdelendFL \DIFaddbeginFL \DIFaddFL{(Missing) }\DIFaddendFL & \DIFdelbeginFL \DIFdelFL{2                }\DIFdelendFL \DIFaddbeginFL \DIFaddFL{10 }& \DIFaddFL{7.7 }& \DIFaddFL{25 }& \DIFaddFL{19.2}\DIFaddendFL \\
\DIFdelbeginFL %DIFDELCMD < \quad %%%
\DIFdelFL{No Response        }\DIFdelendFL \DIFaddbeginFL \DIFaddFL{Household size }\DIFaddendFL & \DIFaddbeginFL \DIFaddFL{1 }& \DIFaddFL{0 }& \DIFaddFL{0.0 }& \DIFaddFL{4 }& \DIFaddFL{3.1}\\
 & \DIFaddFL{2 }& \DIFaddFL{0 }& \DIFaddFL{0.0 }& \DIFaddendFL 10 \DIFdelbeginFL \DIFdelFL{(18.18\%}\DIFdelendFL \DIFaddbeginFL & \DIFaddFL{7.7}\\
 & \DIFaddFL{3 }& \DIFaddFL{0 }& \DIFaddFL{0.0 }& \DIFaddFL{7 }& \DIFaddFL{5.4}\\
 & \DIFaddFL{4+ }& \DIFaddFL{0 }& \DIFaddFL{0.0 }& \DIFaddFL{29 }& \DIFaddFL{22.3}\\
 & \DIFaddFL{(Missing}\DIFaddendFL ) & \DIFaddbeginFL \DIFaddFL{55 }& \DIFaddFL{42.3 }& \DIFaddendFL 25 \DIFdelbeginFL \DIFdelFL{(33.33\%) }\DIFdelendFL \DIFaddbeginFL & \DIFaddFL{19.2}\DIFaddendFL \\
\DIFdelbeginFL \emph{\DIFdelFL{Auto Availability}} %DIFAUXCMD
\DIFdelendFL \DIFaddbeginFL \DIFaddFL{Age }\DIFaddendFL & \DIFaddbeginFL \DIFaddFL{Under 18 }\DIFaddendFL & \DIFaddbeginFL \DIFaddFL{0 }& \DIFaddFL{0.0 }& \DIFaddFL{3 }& \DIFaddFL{2.3}\DIFaddendFL \\
 \DIFdelbeginFL %DIFDELCMD < \quad %%%
\DIFdelendFL \DIFaddbeginFL & \DIFaddFL{18-24 }& \DIFaddFL{12 }& \DIFaddFL{9.2 }& \DIFaddFL{8 }& \DIFaddFL{6.2}\\
 & \DIFaddFL{25-44 }& \DIFaddFL{24 }& \DIFaddFL{18.5 }& \DIFaddFL{28 }& \DIFaddFL{21.5}\\
 & \DIFaddFL{45-64 }& \DIFaddFL{9 }& \DIFaddFL{6.9 }& \DIFaddFL{10 }& \DIFaddFL{7.7}\\
 & \DIFaddFL{Over 65 }& \DIFaddendFL 0 & \DIFaddbeginFL \DIFaddFL{0.0 }& \DIFaddFL{1 }& \DIFaddFL{0.8}\\
 & \DIFaddFL{(Missing) }& \DIFaddFL{10 }& \DIFaddFL{7.7 }& \DIFaddFL{25 }& \DIFaddFL{19.2}\\
\DIFaddFL{Auto availability }& \DIFaddendFL 0 & \DIFaddbeginFL \DIFaddFL{0 }& \DIFaddFL{0.0 }& \DIFaddendFL 4 \DIFaddbeginFL & \DIFaddFL{3.1}\DIFaddendFL \\
 \DIFdelbeginFL %DIFDELCMD < \quad %%%
\DIFdelendFL \DIFaddbeginFL & \DIFaddendFL 1 & 18 & \DIFaddbeginFL \DIFaddFL{13.8 }& \DIFaddendFL 19 \DIFaddbeginFL & \DIFaddFL{14.6}\DIFaddendFL \\
 \DIFdelbeginFL %DIFDELCMD < \quad %%%
\DIFdelendFL \DIFaddbeginFL & \DIFaddendFL 2 & 13 & \DIFaddbeginFL \DIFaddFL{10.0 }& \DIFaddendFL 18 \DIFaddbeginFL & \DIFaddFL{13.8}\DIFaddendFL \\
 \DIFdelbeginFL %DIFDELCMD < \quad %%%
\DIFdelendFL \DIFaddbeginFL & \DIFaddendFL 3 & 8 & \DIFaddbeginFL \DIFaddFL{6.2 }& \DIFaddendFL 8 \DIFaddbeginFL & \DIFaddFL{6.2}\DIFaddendFL \\
 \DIFdelbeginFL %DIFDELCMD < \quad %%%
\DIFdelendFL \DIFaddbeginFL & \DIFaddendFL 4+ & 3 & \DIFaddbeginFL \DIFaddFL{2.3 }& \DIFaddendFL 5 \DIFaddbeginFL & \DIFaddFL{3.8}\DIFaddendFL \\
 \DIFdelbeginFL %DIFDELCMD < \quad %%%
\DIFdelFL{No Response        }\DIFdelendFL & \DIFdelbeginFL \DIFdelFL{13 (23.64\%}\DIFdelendFL \DIFaddbeginFL \DIFaddFL{(Missing}\DIFaddendFL ) & \DIFdelbeginFL \DIFdelFL{21 (28.00\%)     }%DIFDELCMD < \\
%DIFDELCMD < %%%
\emph{\DIFdelFL{Income}}             %DIFAUXCMD
\DIFdelendFL \DIFaddbeginFL \DIFaddFL{13 }\DIFaddendFL & \DIFaddbeginFL \DIFaddFL{10.0 }\DIFaddendFL & \DIFaddbeginFL \DIFaddFL{21 }& \DIFaddFL{16.2}\DIFaddendFL \\
\DIFdelbeginFL %DIFDELCMD < \quad %%%
\DIFdelFL{Less than }\DIFdelendFL \DIFaddbeginFL \DIFaddFL{Income }& \DIFaddFL{Less than \textbackslash{}}\DIFaddendFL \$44,999 & 0 & \DIFaddbeginFL \DIFaddFL{0.0 }& \DIFaddendFL 8 \DIFaddbeginFL & \DIFaddFL{6.2}\DIFaddendFL \\
 \DIFdelbeginFL %DIFDELCMD < \quad %%%
\DIFdelendFL \DIFaddbeginFL & \DIFaddFL{\textbackslash{}}\DIFaddendFL \$45,000 \DIFdelbeginFL \DIFdelFL{- \$99,999  }\DIFdelendFL \DIFaddbeginFL \DIFaddFL{to \textbackslash{}\$100,000 }\DIFaddendFL & 0 & \DIFaddbeginFL \DIFaddFL{0.0 }& \DIFaddendFL 17 \DIFaddbeginFL & \DIFaddFL{13.1}\DIFaddendFL \\
 \DIFdelbeginFL %DIFDELCMD < \quad %%%
\DIFdelFL{Over \$100k        }\DIFdelendFL \DIFaddbeginFL & \DIFaddFL{Over \textbackslash{}\$100,000 }\DIFaddendFL & 0 & \DIFaddbeginFL \DIFaddFL{0.0 }& \DIFaddendFL 17 \DIFaddbeginFL & \DIFaddFL{13.1}\DIFaddendFL \\
 \DIFdelbeginFL %DIFDELCMD < \quad %%%
\DIFdelFL{No Response        }\DIFdelendFL & \DIFdelbeginFL \DIFdelFL{55 (100.00\%}\DIFdelendFL \DIFaddbeginFL \DIFaddFL{(Missing}\DIFaddendFL ) & \DIFdelbeginFL \DIFdelFL{31 (41.33\%)     }%DIFDELCMD < \\
%DIFDELCMD < %%%
\emph{\DIFdelFL{Age}}                %DIFAUXCMD
\DIFdelendFL \DIFaddbeginFL \DIFaddFL{55 }\DIFaddendFL & \DIFaddbeginFL \DIFaddFL{42.3 }\DIFaddendFL & \DIFaddbeginFL \DIFaddFL{33 }& \DIFaddFL{25.4}\DIFaddendFL \\
\DIFdelbeginFL %DIFDELCMD < \quad %%%
\DIFdelFL{Under 18           }\DIFdelendFL \DIFaddbeginFL \DIFaddFL{Weekly transit use }\DIFaddendFL & \DIFdelbeginFL \DIFdelFL{0                 }\DIFdelendFL \DIFaddbeginFL \DIFaddFL{Five days or more }\DIFaddendFL & \DIFdelbeginFL \DIFdelFL{3                }%DIFDELCMD < \\
%DIFDELCMD < \quad %%%
\DIFdelFL{18-24              }\DIFdelendFL \DIFaddbeginFL \DIFaddFL{25 }\DIFaddendFL & \DIFdelbeginFL \DIFdelFL{12                }\DIFdelendFL \DIFaddbeginFL \DIFaddFL{19.2 }\DIFaddendFL & \DIFdelbeginFL \DIFdelFL{8                }\DIFdelendFL \DIFaddbeginFL \DIFaddFL{25 }& \DIFaddFL{19.2}\DIFaddendFL \\
 \DIFdelbeginFL %DIFDELCMD < \quad %%%
\DIFdelFL{25-44              }\DIFdelendFL & \DIFdelbeginFL \DIFdelFL{24                }\DIFdelendFL \DIFaddbeginFL \DIFaddFL{One day or less frequently }\DIFaddendFL & \DIFdelbeginFL \DIFdelFL{28               }%DIFDELCMD < \\
%DIFDELCMD < \quad %%%
\DIFdelFL{45-64              }\DIFdelendFL \DIFaddbeginFL \DIFaddFL{8 }\DIFaddendFL & \DIFdelbeginFL \DIFdelFL{9                 }\DIFdelendFL \DIFaddbeginFL \DIFaddFL{6.2 }\DIFaddendFL & \DIFdelbeginFL \DIFdelFL{10               }\DIFdelendFL \DIFaddbeginFL \DIFaddFL{13 }& \DIFaddFL{10.0}\DIFaddendFL \\
 \DIFdelbeginFL %DIFDELCMD < \quad %%%
\DIFdelFL{Over 65            }\DIFdelendFL & \DIFdelbeginFL \DIFdelFL{0                 }\DIFdelendFL \DIFaddbeginFL \DIFaddFL{Two to four days }\DIFaddendFL & \DIFdelbeginFL \DIFdelFL{1                }%DIFDELCMD < \\
%DIFDELCMD < \quad %%%
\DIFdelFL{No Response        }\DIFdelendFL \DIFaddbeginFL \DIFaddFL{22 }\DIFaddendFL & \DIFdelbeginFL \DIFdelFL{10 (18.18\%)      }\DIFdelendFL \DIFaddbeginFL \DIFaddFL{16.9 }\DIFaddendFL & \DIFdelbeginFL \DIFdelFL{25 (33.33\%)     }\DIFdelendFL \DIFaddbeginFL \DIFaddFL{37 }& \DIFaddFL{28.5}\DIFaddendFL \\
\bottomrule
\end{tabular}
\end{table}

A primary motivation for the survey was to understand awareness of the
microtransit service among UTA transit riders. In the \DIFdelbegin \DIFdel{“Before” }\DIFdelend \DIFaddbegin \DIFadd{``Before'' }\DIFaddend tranche, only 6
of the 55 respondents (11\%) stated they had previously heard of the system. Of
the 58 interviews in the \DIFdelbegin \DIFdel{“After” }\DIFdelend \DIFaddbegin \DIFadd{``After'' }\DIFaddend tranche not conducted on the microtransit
service, 34 (59\%) had previously heard of the service. This increase in general
awareness of the system indicates both that the UTA marketing efforts were
effective, and also that the responses to the subsequent question of likeliness
to use the service are based in some level of understanding.

Figure \ref{fig:likelihood} shows the reported likelihood of survey respondents
to download the necessary application and use the microtransit service,
separated by access mode. Respondents who were already using the service
selected \DIFdelbegin \DIFdel{“}\DIFdelend \DIFaddbegin \DIFadd{``}\DIFaddend 5: Extremely Likely.\DIFdelbegin \DIFdel{” }\DIFdelend \DIFaddbegin \DIFadd{'' }\DIFaddend The first result of this analysis is that there
appears to be a polarization in opinions after the service commenced operations.
Although there are some strong feelings against and for the service in the
\DIFdelbegin \DIFdel{“Before” }\DIFdelend \DIFaddbegin \DIFadd{``Before'' }\DIFaddend tranche, the neutral opinions have comparatively disappeared in the
\DIFdelbegin \DIFdel{“After” }\DIFdelend \DIFaddbegin \DIFadd{``After'' }\DIFaddend tranche. This likely reflects the increasing awareness of the service
discussed above and a hardening of engrained or newly learned habits. It is
important also to stress that the question will not necessarily elicit an
opinion as to whether the service should exist, merely whether the particular
respondent is willing to use it.

The sample is too small to conduct meaningful statistical inference on the role
that access mode plays in these opinions, but some discussion of these
observations is still worthwhile. The apparent turning of bicycle users against
the service is likely statistical noise, though it should also be noted that the
\DIFdelbegin \DIFdel{“After” }\DIFdelend \DIFaddbegin \DIFadd{``After'' }\DIFaddend tranche was collected in January and February, when Utah is typically
cold with snow on the ground. Perhaps individuals who are still cycling at those
times will persist in doing so. It is also interesting to note that there
appears to be little overall correlation between access mode and expressed
willingness to use the service, unless the UTA On Demand service attracts people
who would not have used the service otherwise. Of these individuals who
responded to a question about their hypothetical alternative mode, four reported
that they would have used a Transportation Network Company (TNC; e.g.\DIFaddbegin \DIFadd{~}\DIFaddend Uber,
Lyft, etc.), two would have used regular UTA services, two would have driven to
the transit station, one would have walked, and one would not have used transit
at all. Additionally, the text responses to the access mode question in the
\DIFdelbegin \DIFdel{“before” }\DIFdelend \DIFaddbegin \DIFadd{``before'' }\DIFaddend tranche revealed a number of individuals who used a TNC to access the
system. This supplies anecdotal evidence that microtransit is competing more
against commercial TNC offerings than against conventional transit services.

\begin{figure}
\DIFdelbeginFL %DIFDELCMD < \centering
%DIFDELCMD <     \includegraphics[width = \textwidth]{access_mode.eps}
%DIFDELCMD <     %%%
\DIFdelendFL \DIFaddbeginFL \includegraphics[width=0.8\linewidth]{via_riders_files/figure-latex/likelihood-1} \DIFaddendFL \caption{Reported likelihood of using microtransit by transit access mode.}\label{fig:likelihood}
\end{figure}

The next consideration is whether the expressed or observed likeliness to use
the microtransit service is related to the demographic characteristics of the
respondents. Noting the low response rate to many of the demographic questions
(see Table \DIFdelbegin \DIFdel{\ref{tab:survey-summary}}\DIFdelend \DIFaddbegin \DIFadd{\ref{tab:survey-respondents}}\DIFaddend ), it is not possible to construct a model
that would predict the likeliness score as a function of these characteristics
in combination. It is still valuable, however, to consider how the observed
distribution of these characteristics differs between individuals who are or are
not likely to use the service. These distributions are shown in Table
\DIFdelbegin \DIFdel{\ref{tab:likeliness}}\DIFdelend \DIFaddbegin \DIFadd{\ref{tab:fisher-table}}\DIFaddend , along with the result of a two-sided Fisher exact test of
independence between the indicated characteristic distribution and the
three-category likeliness response. In this test the null hypothesis is that the
two distributions are independent with the alternative being there is some
dependence between the characteristic and the response. A \DIFdelbegin \DIFdel{$p$}\DIFdelend \DIFaddbegin \DIFadd{\(p\)}\DIFaddend -value less than a
given critical threshold indicates that the null hypothesis has a low
probability and should be rejected. A conventional value of the critical value
is \DIFdelbegin \DIFdel{$\alpha=0.05$}\DIFdelend \DIFaddbegin \DIFadd{\(\alpha=0.05\)}\DIFaddend , though given the small sample sizes in this survey other
critical values may be suggestive of the need for future evaluation.

\DIFdelbegin %DIFDELCMD < \begin{table}[ht]
%DIFDELCMD <     %%%
\DIFdelendFL \DIFaddbeginFL \begin{table}

\caption{\label{tab:fisher-table}\DIFaddFL{Distribution of Rider Characteristics by Reported Likeliness}}
\DIFaddendFL \centering
\DIFdelbeginFL %DIFDELCMD < \renewcommand{\arraystretch}{1.5}
%DIFDELCMD <     %%%
%DIFDELCMD < \caption{%
{%DIFAUXCMD
\DIFdelFL{Distribution of Rider Characteristics by Reported Likeliness}}
    %DIFAUXCMD
%DIFDELCMD < \label{tab:likeliness}
%DIFDELCMD < \begin{tabular}{@{}lccc@{}}
%DIFDELCMD < %%%
\DIFdelendFL \DIFaddbeginFL \begin{tabular}[t]{llrrr}
\DIFaddendFL \toprule
 & \DIFdelbeginFL \DIFdelFL{Not Likely (1 and 2) }\DIFdelendFL \DIFaddbeginFL \DIFaddFL{Demographic }& \DIFaddFL{Not Likely }\DIFaddendFL & Neutral \DIFdelbeginFL \DIFdelFL{(3) }\DIFdelendFL & Likely\DIFdelbeginFL \DIFdelFL{(4 and 5) }\DIFdelendFL \\
\midrule
\DIFdelbeginFL %DIFDELCMD < \multicolumn{4}{l}{\emph{Smartphone} ($p_F = 0.563$ on 2 degrees of freedom)}%%%
\DIFdelendFL \DIFaddbeginFL \addlinespace[0.3em]
\multicolumn{5}{l}{\textbf{Smartphone; Fisher p-value: 0.5633}}\DIFaddendFL \\
\DIFdelbeginFL %DIFDELCMD < \quad %%%
\DIFdelFL{Yes                }\DIFdelendFL \DIFaddbeginFL \DIFaddFL{\hspace{1em} }\DIFaddendFL & \DIFdelbeginFL \DIFdelFL{41 (95\%)            }%DIFDELCMD < & %%%
\DIFdelFL{8 (89\%)    }%DIFDELCMD < & %%%
\DIFdelFL{30 (97\%)        }%DIFDELCMD < \\
%DIFDELCMD < \quad %%%
\DIFdelendFL No & 2 \DIFdelbeginFL \DIFdelFL{(5\%)              }\DIFdelendFL & 1 \DIFdelbeginFL \DIFdelFL{(11\%)    }\DIFdelendFL & 1\DIFdelbeginFL \DIFdelFL{(3\%)          }\DIFdelendFL \\
\DIFdelbeginFL %DIFDELCMD < \multicolumn{4}{l}{\emph{Household Size} ($p_F = 0.207$ on 6 degrees of freedom)}%%%
\DIFdelendFL \DIFaddbeginFL 

\DIFaddFL{\hspace{1em} }& \DIFaddFL{Yes }& \DIFaddFL{41 }& \DIFaddFL{8 }& \DIFaddFL{30}\DIFaddendFL \\
\DIFdelbeginFL %DIFDELCMD < \quad %%%
\DIFdelendFL \DIFaddbeginFL 

\addlinespace[0.3em]
\multicolumn{5}{l}{\textbf{Household Size; Fisher p-value: 0.2068}}\\
\DIFaddFL{\hspace{1em} }& \DIFaddendFL 1 & 2 \DIFdelbeginFL \DIFdelFL{(7\%)              }\DIFdelendFL & 0 \DIFdelbeginFL \DIFdelFL{(0\%)     }\DIFdelendFL & 2\DIFdelbeginFL \DIFdelFL{(14\%)         }\DIFdelendFL \\
\DIFdelbeginFL %DIFDELCMD < \quad %%%
\DIFdelendFL \DIFaddbeginFL 

\DIFaddFL{\hspace{1em} }& \DIFaddendFL 2 & 8 \DIFdelbeginFL \DIFdelFL{(29\%)             }\DIFdelendFL & 0 \DIFdelbeginFL \DIFdelFL{(0\%)     }\DIFdelendFL & 1\DIFdelbeginFL \DIFdelFL{(7\%)          }\DIFdelendFL \\
\DIFdelbeginFL %DIFDELCMD < \quad %%%
\DIFdelendFL \DIFaddbeginFL 

\DIFaddFL{\hspace{1em} }& \DIFaddendFL 3 & 4 \DIFdelbeginFL \DIFdelFL{(14\%)             }\DIFdelendFL & 1 \DIFdelbeginFL \DIFdelFL{(25\%)    }\DIFdelendFL & 0\DIFdelbeginFL \DIFdelFL{(0\%)          }\DIFdelendFL \\
\DIFdelbeginFL %DIFDELCMD < \quad %%%
\DIFdelendFL \DIFaddbeginFL 

\DIFaddFL{\hspace{1em} }& \DIFaddendFL 4+ & 14 \DIFdelbeginFL \DIFdelFL{(50\%)            }\DIFdelendFL & 3 \DIFdelbeginFL \DIFdelFL{(75\%)    }\DIFdelendFL & 11\DIFdelbeginFL \DIFdelFL{(79\%)        }\DIFdelendFL \\
\DIFdelbeginFL %DIFDELCMD < \multicolumn{4}{l}{\emph{Auto Availability} ($p_F = 0.659$ on 8 degrees of freedom)}%%%
\DIFdelendFL \DIFaddbeginFL 

\addlinespace[0.3em]
\multicolumn{5}{l}{\textbf{Auto Availablity; Fisher p-value: 0.6593}}\DIFaddendFL \\
\DIFdelbeginFL %DIFDELCMD < \quad %%%
\DIFdelendFL \DIFaddbeginFL \DIFaddFL{\hspace{1em} }& \DIFaddendFL 0 & 1 \DIFdelbeginFL \DIFdelFL{(2\%)              }\DIFdelendFL & 0 \DIFdelbeginFL \DIFdelFL{(0\%)     }\DIFdelendFL & 3\DIFdelbeginFL \DIFdelFL{(10\%)         }\DIFdelendFL \\
\DIFdelbeginFL %DIFDELCMD < \quad %%%
\DIFdelendFL \DIFaddbeginFL 

\DIFaddFL{\hspace{1em} }& \DIFaddendFL 1 & 22 \DIFdelbeginFL \DIFdelFL{(49\%)            }\DIFdelendFL & 3 \DIFdelbeginFL \DIFdelFL{(33\%)    }\DIFdelendFL & 10\DIFdelbeginFL \DIFdelFL{(33\%)        }\DIFdelendFL \\
\DIFdelbeginFL %DIFDELCMD < \quad %%%
\DIFdelendFL \DIFaddbeginFL 

\DIFaddFL{\hspace{1em} }& \DIFaddendFL 2 & 12 \DIFdelbeginFL \DIFdelFL{(27\%)            }\DIFdelendFL & 3 \DIFdelbeginFL \DIFdelFL{(33\%)    }\DIFdelendFL & 9\DIFdelbeginFL \DIFdelFL{(30\%)         }\DIFdelendFL \\
\DIFdelbeginFL %DIFDELCMD < \quad %%%
\DIFdelendFL \DIFaddbeginFL 

\DIFaddFL{\hspace{1em} }& \DIFaddendFL 3 & 7 \DIFdelbeginFL \DIFdelFL{(16\%)             }\DIFdelendFL & 1 \DIFdelbeginFL \DIFdelFL{(11\%)    }\DIFdelendFL & 5\DIFdelbeginFL \DIFdelFL{(17\%)         }\DIFdelendFL \\
\DIFdelbeginFL %DIFDELCMD < \quad %%%
\DIFdelendFL \DIFaddbeginFL 

\DIFaddFL{\hspace{1em} }& \DIFaddendFL 4+ & 3 \DIFdelbeginFL \DIFdelFL{(7\%)              }\DIFdelendFL & 2 \DIFdelbeginFL \DIFdelFL{(22\%)    }\DIFdelendFL & 3\DIFdelbeginFL \DIFdelFL{(10\%)         }\DIFdelendFL \\
\DIFdelbeginFL %DIFDELCMD < \multicolumn{4}{l}{\emph{Income} ($p_F = 0.687$ on 4 degrees of freedom)}%%%
\DIFdelendFL \DIFaddbeginFL 

\addlinespace[0.3em]
\multicolumn{5}{l}{\textbf{Income; Fisher p-value: 0.6873}}\DIFaddendFL \\
\DIFdelbeginFL %DIFDELCMD < \quad %%%
\DIFdelFL{Less than }\DIFdelendFL \DIFaddbeginFL \DIFaddFL{\hspace{1em} }& \DIFaddFL{Less than \textbackslash{}}\DIFaddendFL \$44,999 & 4 \DIFdelbeginFL \DIFdelFL{(17\%)             }\DIFdelendFL & 1 \DIFdelbeginFL \DIFdelFL{(33\%)    }\DIFdelendFL & 3\DIFdelbeginFL \DIFdelFL{(21\%)         }\DIFdelendFL \\
\DIFdelbeginFL %DIFDELCMD < \quad %%%
\DIFdelFL{$45,000 - $99,999  }\DIFdelendFL \DIFaddbeginFL 

\DIFaddFL{\hspace{1em} }\DIFaddendFL & \DIFaddbeginFL \DIFaddFL{\textbackslash{}\$45,000 to \textbackslash{}\$100,000 }& \DIFaddendFL 10 \DIFdelbeginFL \DIFdelFL{(44\%)            }\DIFdelendFL & 2 \DIFdelbeginFL \DIFdelFL{(67\%)    }\DIFdelendFL & 5\DIFdelbeginFL \DIFdelFL{(36\%)         }\DIFdelendFL \\
\DIFdelbeginFL %DIFDELCMD < \quad %%%
\DIFdelFL{Over \$100k        }\DIFdelendFL \DIFaddbeginFL 

\DIFaddFL{\hspace{1em} }& \DIFaddFL{Over \textbackslash{}\$100,000 }\DIFaddendFL & 9 \DIFdelbeginFL \DIFdelFL{(39\%)             }\DIFdelendFL & 0 \DIFdelbeginFL \DIFdelFL{(0\%)     }\DIFdelendFL & 6\DIFdelbeginFL \DIFdelFL{(43\%)         }\DIFdelendFL \\
\DIFdelbeginFL %DIFDELCMD < \quad %%%
\DIFdelendFL \DIFaddbeginFL 

\addlinespace[0.3em]
\multicolumn{5}{l}{\textbf{Age; Fisher p-value: 0.0036}}\\
\DIFaddFL{\hspace{1em} }& \DIFaddendFL Under 18 & 1 \DIFdelbeginFL \DIFdelFL{(2\%)              }\DIFdelendFL & 2 \DIFdelbeginFL \DIFdelFL{(20\%)    }\DIFdelendFL & 0\DIFdelbeginFL \DIFdelFL{(0\%)          }\DIFdelendFL \\
\DIFdelbeginFL %DIFDELCMD < \multicolumn{4}{l}{\emph{Age} ($p_F = 0.00364^*$ on 8 degrees of freedom)}\\
%DIFDELCMD < \quad %%%
\DIFdelendFL \DIFaddbeginFL 

\DIFaddFL{\hspace{1em} }& \DIFaddendFL 18-24 & 7 \DIFdelbeginFL \DIFdelFL{(16\%)             }\DIFdelendFL & 2 \DIFdelbeginFL \DIFdelFL{(20\%)    }\DIFdelendFL & 9\DIFdelbeginFL \DIFdelFL{(31\%)         }\DIFdelendFL \\
\DIFdelbeginFL %DIFDELCMD < \quad %%%
\DIFdelendFL \DIFaddbeginFL 

\DIFaddFL{\hspace{1em} }& \DIFaddendFL 25-44 & 28 \DIFdelbeginFL \DIFdelFL{(64\%)            }\DIFdelendFL & 1 \DIFdelbeginFL \DIFdelFL{(10\%)    }\DIFdelendFL & 17\DIFdelbeginFL \DIFdelFL{(59\%)        }\DIFdelendFL \\
\DIFdelbeginFL %DIFDELCMD < \quad %%%
\DIFdelendFL \DIFaddbeginFL 

\DIFaddFL{\hspace{1em} }& \DIFaddendFL 45-64 & 7 \DIFdelbeginFL \DIFdelFL{(16\%)             }\DIFdelendFL & 5 \DIFdelbeginFL \DIFdelFL{(50\%)    }\DIFdelendFL & 3\DIFdelbeginFL \DIFdelFL{(10\%)         }\DIFdelendFL \\
\DIFdelbeginFL %DIFDELCMD < \quad %%%
\DIFdelendFL \DIFaddbeginFL 

\DIFaddFL{\hspace{1em} }& \DIFaddendFL Over 65 & 1 \DIFdelbeginFL \DIFdelFL{(2\%)              }\DIFdelendFL & 0 \DIFdelbeginFL \DIFdelFL{(0\%)     }\DIFdelendFL & 0\DIFdelbeginFL \DIFdelFL{(0\%)         }\DIFdelendFL \\
\bottomrule
\DIFdelbeginFL %DIFDELCMD < \multicolumn{4}{l}{$^*$ indicates $p$-value less than 0.05}
%DIFDELCMD < 

%DIFDELCMD < %%%
\DIFdelendFL \end{tabular}
\end{table}

Smartphone use appears to not be a contributing factor in the likeliness of
using microtransit, as almost all respondents use a smartphone regardless of
their reported likeliness. We also fail to reject the null hypothesis of
independence between the likeliness to use microtransit and both auto
availability and household income. The joint distribution of reported likeliness
and household size suggests there could be some dependence, with members of
smaller households more frequently expressing reluctance to use microtransit.
This finding, if it could be verified, would be somewhat counter to the \emph{a
priori} expectations of UTA. A Fisher test of independence between these
household size and expressed likeliness still fails to conclusively reject the
null hypothesis but given the small sample size and counter-intuitive results,
future investigation is warranted. This is particularly true given that
automobile availability and household size go hand-in-hand: a household with
more individuals, particularly driving-age individuals, will be more constrained
in their driving behavior even with multiple household automobiles. Considering
these two variables together will be important for future research but cannot be
attempted here.

A clear statistical result is shown, however, between the reported willingness
to use microtransit and the age of the respondent. This significant result
persists when we recombine the age categories as well as discard neutral
responses. Table \ref{tab:age-difference} shows the differences between the
observed values in the joint distribution of these two variables and the
expected values based on the marginal distributions were the two variables to be
completely independent. The largest differences occur in three noticeable
places. First, individuals in the 18-24 years old category are more likely to
express willingness to use microtransit. Second, individuals between 45 and 64
are more likely to express a neutral opinion than a positive or strictly
unlikely one. Finally, individuals between 25 and 44 are \DIFdelbegin \DIFdel{– perhaps surprisingly
– }\DIFdelend \DIFaddbegin \DIFadd{-- perhaps surprisingly
-- }\DIFaddend substantially more likely to express a negative opinion than a neutral one;
these individuals are also modestly more likely than expected to express
positive willingness to use transit.

\DIFdelbegin %DIFDELCMD < \begin{table}[ht]
%DIFDELCMD <     %%%
\DIFdelendFL \DIFaddbeginFL \begin{table}

\caption{\label{tab:age-difference}\DIFaddFL{Difference of observed and expected frequencies for age and likeliness}}
\DIFaddendFL \centering
\DIFdelbeginFL %DIFDELCMD < \renewcommand{\arraystretch}{1.5}
%DIFDELCMD <     %%%
%DIFDELCMD < \caption{%
{%DIFAUXCMD
\DIFdelFL{Difference of Observed and Expected Frequencies for Age and Likeliness}}
    %DIFAUXCMD
%DIFDELCMD < \label{tab:age-difference}
%DIFDELCMD <     \begin{tabular}{@{}lccc@{}}
%DIFDELCMD < %%%
\DIFdelendFL \DIFaddbeginFL \begin{tabular}[t]{lrrr}
\DIFaddendFL \toprule
  \DIFdelbeginFL \DIFdelFL{Age      }\DIFdelendFL & Not Likely & Neutral & Likely\\
\midrule
Under 18 & \DIFdelbeginFL \DIFdelFL{– 0.59     }\DIFdelendFL \DIFaddbeginFL \DIFaddFL{-0.5904 }\DIFaddendFL & \DIFdelbeginFL \DIFdelFL{1.64    }\DIFdelendFL \DIFaddbeginFL \DIFaddFL{1.6386 }\DIFaddendFL & \DIFdelbeginFL \DIFdelFL{– 1.05 }\DIFdelendFL \DIFaddbeginFL \DIFaddFL{-1.0482}\DIFaddendFL \\
18-24 & \DIFdelbeginFL \DIFdelFL{– 2.54     }\DIFdelendFL \DIFaddbeginFL \DIFaddFL{-2.5422 }\DIFaddendFL & \DIFdelbeginFL \DIFdelFL{– 0.17  }\DIFdelendFL \DIFaddbeginFL \DIFaddFL{-0.1687 }\DIFaddendFL & \DIFdelbeginFL \DIFdelFL{2.71   }\DIFdelendFL \DIFaddbeginFL \DIFaddFL{2.7108}\DIFaddendFL \\
25-44 & \DIFdelbeginFL \DIFdelFL{3.61       }\DIFdelendFL \DIFaddbeginFL \DIFaddFL{3.6145 }\DIFaddendFL & \DIFdelbeginFL \DIFdelFL{– 4.54  }\DIFdelendFL \DIFaddbeginFL \DIFaddFL{-4.5422 }\DIFaddendFL & \DIFdelbeginFL \DIFdelFL{0.93   }\DIFdelendFL \DIFaddbeginFL \DIFaddFL{0.9277}\DIFaddendFL \\
45-64 & \DIFdelbeginFL \DIFdelFL{–0.95      }\DIFdelendFL \DIFaddbeginFL \DIFaddFL{-0.9518 }\DIFaddendFL & \DIFdelbeginFL \DIFdelFL{3.19    }\DIFdelendFL \DIFaddbeginFL \DIFaddFL{3.1928 }\DIFaddendFL & \DIFdelbeginFL \DIFdelFL{– 2.24 }\DIFdelendFL \DIFaddbeginFL \DIFaddFL{-2.2410}\DIFaddendFL \\
Over 65 & \DIFdelbeginFL \DIFdelFL{0.47       }\DIFdelendFL \DIFaddbeginFL \DIFaddFL{0.4699 }\DIFaddendFL & \DIFdelbeginFL \DIFdelFL{– 0.12  }\DIFdelendFL \DIFaddbeginFL \DIFaddFL{-0.1205 }\DIFaddendFL & \DIFdelbeginFL \DIFdelFL{– 0.35}\DIFdelendFL \DIFaddbeginFL \DIFaddFL{-0.3494}\DIFaddendFL \\
\bottomrule
\end{tabular}
\end{table}

%DIF < %%%%%%%%%%%%%%%%%%%%%%%%%%%%%%%%%%%%%%%%%
\DIFdelbegin \section{\DIFdel{Discussion}}
%DIFAUXCMD
\addtocounter{section}{-1}%DIFAUXCMD
\DIFdel{We }\DIFdelend \DIFaddbegin \hypertarget{multiple-imputation}{%
\subsection{Multiple Imputation}\label{multiple-imputation}}

\DIFadd{The attitudinal survey analysis presented by \mbox{%DIFAUXCMD
\citet{KONIG2020954} }\hspace{0pt}%DIFAUXCMD
included an
unreported fraction of incomplete records that were filled via hot-deck
imputation. The applicability of missing variable imputation to the present situation
is potentially inappropriate, given both the degree of missingness in some
variables as well as the potential for these variables to not be missing at
random. Nevertheless, the suggestive results on the age, frequency of transit
use, and household size variable indicates that such an analysis might be
illuminating, if not conclusive.
}

\DIFadd{We generated 20 multiple imputed complete datasets with the following
variables:
}

\begin{itemize}
\tightlist
\item
  \DIFadd{Available autos
}\item
  \DIFadd{Age
}\item
  \DIFadd{Household Size
}\item
  \DIFadd{Income
}\item
  \DIFadd{Frequency of Using UTA (as categories)
}\end{itemize}

\DIFadd{The datasets were generated via chained equations using classification and
regression tree methods encoded in the }\texttt{\DIFadd{mice}} \DIFadd{package for R \mbox{%DIFAUXCMD
\citep{mice2011, R-base}}\hspace{0pt}%DIFAUXCMD
. We
are primarily interested in exploring the validity of the results we had already
that had been suggested with the incomplete dataset: specifically, the relationship
between reported likeliness and age, household size, and frequency of using UTA.
We computed the Fisher exact test of an independent relationship between these
variables on each imputed dataset. The Fisher test does not have a test
statistic and generates \(p\)-values via simulation; we therefore pool the
\(p\)-values from the 20 tests using a methodology proposed by
\mbox{%DIFAUXCMD
\citet{LichtThesis}}\hspace{0pt}%DIFAUXCMD
.
}

\DIFadd{The pooled \(p\)-values obtained through the multiply imputed data sets are given
in Table \ref{tab:mi-fishers-table} alongside the original, non-imputed
\(p\)-values. In the imputed data, there appears to be a clearer relationship
between likeliness and household size, sufficient to reject the null hypothesis
of independence at a standard significance level. The relationship between age
and likeliness is weakened, but still would be considered significant. There
remains no evidence that expressed likeliness to use microtransit is correlated
in any way with the frequency of using transit.
}

\begin{table}

\caption{\label{tab:mi-fishers-table}\DIFaddFL{Fisher Exact p-value with Imputation}}
\centering
\begin{tabular}[t]{lrr}
\toprule
\DIFaddFL{Variable }& \DIFaddFL{Original p }& \DIFaddFL{Imputed p}\\
\midrule
\DIFaddFL{Household size }& \DIFaddFL{0.2068077 }& \DIFaddFL{0.0374068}\\
\DIFaddFL{Age }& \DIFaddFL{0.0036430 }& \DIFaddFL{0.0131557}\\
\DIFaddFL{Weekly transit use }& \DIFaddFL{0.2937006 }& \DIFaddFL{0.3454466}\\
\bottomrule
\end{tabular}
\end{table}

\hypertarget{discussion}{%
\section{Discussion}\label{discussion}}

\DIFadd{We }\DIFaddend readily acknowledge several limitations of this study, particularly in the
survey design and methodology. The interviews were conducted as a convenience
sample rather than with a rigorous sampling strategy, with the statistical
caveats resulting from that design decision. The sample is also too small to
have substantial statistical power, particularly in statistics calculated on
multiple grouping dimensions. Finally, the survey collected self-reported
responses with no verification or validation of any kind.

Most survey responses were collected on fixed rail transit station platforms.
Passengers waiting for UTA trains were expected to be both more available to
respond to survey questions, as well as more likely to use the microtransit
service than the general population. Additionally, UTA is interested in
supporting its fixed rail transit investments in the service area. There is,
however, no requirement that microtransit passengers use other UTA services;
data supplied by the microtransit provider but not included in this study
suggest that only 58\% percent of microtransit trips began or ended within 500
feet of a UTA rail transit station. This population might have preferences or
patterns that either match or contradict the initial findings of this research.

In discussing the responses to the question of what mode microtransit passengers
would have used were the service not available, we suggested there is anecdotal
evidence that commercial TNC rides are the primary competition. There are still
questions, however, of how use of this service might affect conventional transit
services. Table \ref{tab:uta-ridership} shows the average weekday ridership
during November, December, and January for the period the microtransit service
was operating as well as the same three months in the two prior years
\citet{uta2020boardings}. Total system ridership was remarkably stable during
these three periods. The microtransit service area \DIFdelbegin \DIFdel{– }\DIFdelend \DIFaddbegin \DIFadd{-- }\DIFaddend in this case defined by
ridership on routes F514, 218, 526, F504, F518, F534, F546, and F547 \DIFdelbegin \DIFdel{– }\DIFdelend \DIFaddbegin \DIFadd{-- }\DIFaddend was
declining before the microtransit service began, though the decline accelerated
during the first three months of the service\DIFdelbegin \DIFdel{’}\DIFdelend \DIFaddbegin \DIFadd{'}\DIFaddend s operation. By comparison, the
microtransit service carried approximately 316 passengers per day during its
first three months, more than compensating for the recent observed decline in
transit ridership were this to be identified as a major contributing factor.

\DIFdelbegin %DIFDELCMD < \begin{table}[ht]
%DIFDELCMD <     %%%
\DIFdelendFL \DIFaddbeginFL \begin{table}

\caption{\label{tab:uta-ridership}\DIFaddFL{Average Weekday Ridership, November through January}}
\DIFaddendFL \centering
\DIFdelbeginFL %DIFDELCMD < \renewcommand{\arraystretch}{1.5}
%DIFDELCMD <     %%%
%DIFDELCMD < \caption{%
{%DIFAUXCMD
\DIFdelFL{Average Weekday Ridership, November through January}}
    %DIFAUXCMD
%DIFDELCMD < \label{tab:uta-ridership}
%DIFDELCMD <     \begin{tabular}{@{}lcccc@{}}
%DIFDELCMD < %%%
\DIFdelendFL \DIFaddbeginFL \begin{tabular}[t]{lrrrr}
\DIFaddendFL \toprule
\DIFaddbeginFL \multicolumn{1}{c}{ } \DIFaddendFL & \DIFdelbeginFL %DIFDELCMD < \multicolumn{2}{c}{All   Other UTA Services} %%%
\DIFdelendFL \DIFaddbeginFL \multicolumn{2}{c}{Microtransit Service Area} \DIFaddendFL & \DIFdelbeginFL %DIFDELCMD < \multicolumn{2}{c}{Microtransit   Service Region} %%%
\DIFdelendFL \DIFaddbeginFL \multicolumn{2}{c}{Other UTA Services} \DIFaddendFL \\
\cmidrule(l\DIFaddbeginFL {\DIFaddFL{3pt}}\DIFaddFL{r}{\DIFaddFL{3pt}}\DIFaddendFL ){2-3} \cmidrule(l\DIFaddbeginFL {\DIFaddFL{3pt}}\DIFaddFL{r}{\DIFaddFL{3pt}}\DIFaddendFL ){4-5}
Year & \DIFdelbeginFL \DIFdelFL{Average   Weekday Riders    }\DIFdelendFL \DIFaddbeginFL \DIFaddFL{Avg. Weekday Boardings }\DIFaddendFL & \% Change & \DIFdelbeginFL \DIFdelFL{Average   Weekday Riders    }\DIFdelendFL \DIFaddbeginFL \DIFaddFL{Avg. Weekday Boardings }\DIFaddendFL & \% Change\\
\midrule
\DIFdelbeginFL \DIFdelFL{2019-2018 }\DIFdelendFL \DIFaddbeginFL \DIFaddFL{2017-2018 }\DIFaddendFL & \DIFdelbeginFL \DIFdelFL{147,410                     }\DIFdelendFL \DIFaddbeginFL \DIFaddFL{1179.33 }\DIFaddendFL &  & \DIFdelbeginFL \DIFdelFL{1,179                       }\DIFdelendFL \DIFaddbeginFL \DIFaddFL{147410.0 }\DIFaddendFL & \\
2018-2019 & \DIFdelbeginFL \DIFdelFL{146,743                     }\DIFdelendFL \DIFaddbeginFL \DIFaddFL{1125.00 }\DIFaddendFL & \DIFdelbeginFL \DIFdelFL{–   0.452\%    }\DIFdelendFL \DIFaddbeginFL \DIFaddFL{-4.61 }\DIFaddendFL & \DIFdelbeginFL \DIFdelFL{1,125                       }\DIFdelendFL \DIFaddbeginFL \DIFaddFL{146743.0 }\DIFaddendFL & \DIFdelbeginFL \DIFdelFL{–   4.61\% }\DIFdelendFL \DIFaddbeginFL \DIFaddFL{-0.45}\DIFaddendFL \\
2019-2020 & \DIFdelbeginFL \DIFdelFL{147,010                     }\DIFdelendFL \DIFaddbeginFL \DIFaddFL{970.33 }\DIFaddendFL & \DIFdelbeginFL \DIFdelFL{0.182          }\DIFdelendFL \DIFaddbeginFL \DIFaddFL{-13.75 }\DIFaddendFL & \DIFdelbeginFL \DIFdelFL{970                         }\DIFdelendFL \DIFaddbeginFL \DIFaddFL{147009.8 }\DIFaddendFL & \DIFdelbeginFL \DIFdelFL{–   13.7\% }\DIFdelendFL \DIFaddbeginFL \DIFaddFL{0.18}\DIFaddendFL \\
\bottomrule
\end{tabular}
\end{table}

Another limitation of these findings that is particularly relevant is the onset
of the COVID-19 pandemic. Government-imposed shutdowns and voluntary work
stoppages related to the pandemic did not begin in Utah until the week of March
15th, after data collection for this project had completed. As such, the survey
responses are likely unaffected by changes in behavior related to the pandemic.
However, the pandemic has drastically affected the subsequent operations of both
UTA and Via and is likely to change many of the stated behaviors and attitudes
reported in this study. Many findings of this study will need to be reconsidered
should \DIFdelbegin \DIFdel{“normal” }\DIFdelend \DIFaddbegin \DIFadd{``normal'' }\DIFaddend operations resume.

%DIF < %%%%%%%%%%%%%%%%%%%%%%%%%%%%%%%%%%%%%%%%%
\DIFdelbegin \subsection{\DIFdel{Conclusion}}
%DIFAUXCMD
\addtocounter{subsection}{-1}%DIFAUXCMD
\DIFdelend \DIFaddbegin \hypertarget{conclusion}{%
\subsection{Conclusion}\label{conclusion}}
\DIFaddend 

Microtransit services are regularly put forward as means to support last-mile /
first-mile trips on fixed route transit systems, and several such systems have
been deployed in the recent past. This paper presented initial findings from a
quick response survey aimed at learning who was most willing to use the new
service within weeks of the system launch. These initial findings suggest first
that younger adults are most willing to consider using microtransit services,
and that these services compete most directly with commercial TNC ridehail
offerings rather than conventional transit.

Transit passenger intercept surveys are an important method to determine who is
and who is not using a microtransit service, paired with demographic
characteristics and trip purpose information. To understand the rider
characteristics and trip purposes specifically of microtransit users, by
contrast, better survey methods are needed. In particular, a survey pushed
through the smartphone application used by the passengers would help in reaching
a considerably larger sample. It would also be theoretically possible in that
case for the researchers to pair the survey responses with actual observed trip
patterns for distinct users including origin, destination, and route GPS points,
regularity of use and variance in use patterns, and many other data variables.
Obtaining these data and conducting responsible research with them should be a
priority for the service operators and their agency partners.

%DIF < %%%%%%%%%%%%%%%%%%%%%%%%%%%%%%%%%%%%%%%%%
%DIF >  %%%%%%%%%%%%%%%%%%%%%%%%%%%%%%%%%%%%%%%%%%
%DIF >  %% optional
%DIF >  \supplementary{The following are available online at www.mdpi.com/link, Figure S1: title, Table S1: title, Video S1: title.}
%DIF > 
%DIF >  % Only for the journal Methods and Protocols:
%DIF >  % If you wish to submit a video article, please do so with any other supplementary material.
%DIF >  % \supplementary{The following are available at www.mdpi.com/link: Figure S1: title, Table S1: title, Video S1: title. A supporting video article is available at doi: link.}
\DIFaddbegin 

\DIFaddend \vspace{6pt}

%%%%%%%%%%%%%%%%%%%%%%%%%%%%%%%%%%%%%%%%%%
\DIFdelbegin %DIFDELCMD < \authorcontributions{Data curation, Christian Hunter, Austin Martinez and Elizabeth Smith; Investigation, Christian Hunter, Austin Martinez and Elizabeth Smith; Methodology, Christian Hunter, Austin Martinez and Elizabeth Smith; Project administration, Gregory S. Macfarlane; Supervision, Gregory S. Macfarlane; Writing – original draft, Gregory S. Macfarlane; Writing – review \& editing, Gregory S. Macfarlane, Christian Hunter and Elizabeth Smith.}
%DIFDELCMD < %%%
\DIFdelend \DIFaddbegin \acknowledgments{This project was sponsored by UTA through the BYU Civil Engineering Capstone
Program. The authors would like to thank Jaron Robertson and Shaina Quinn of
UTA and Sahar Shirazi and Kenny Ferrel of WSP for oversight and input
throughout the project.}
\DIFaddend 

%%%%%%%%%%%%%%%%%%%%%%%%%%%%%%%%%%%%%%%%%%
\DIFdelbegin %DIFDELCMD < \funding{This research received no external funding.}
%DIFDELCMD < %%%
\DIFdelend \DIFaddbegin \authorcontributions{Data curation, Christian Hunter, Austin Martinez and Elizabeth Smith;
Investigation, Christian Hunter, Austin Martinez and Elizabeth Smith;
Methodology, Christian Hunter, Austin Martinez and Elizabeth Smith; Project
administration, Gregory S. Macfarlane; Supervision, Gregory S. Macfarlane;
Writing -- original draft, Gregory S. Macfarlane; Writing -- review \& editing,
Gregory S. Macfarlane, Christian Hunter and Elizabeth Smith}
\DIFaddend 

%%%%%%%%%%%%%%%%%%%%%%%%%%%%%%%%%%%%%%%%%%
\DIFdelbegin %DIFDELCMD < \acknowledgments{This project was sponsored by UTA through the BYU Civil Engineering Capstone Program. The authors would like to thank Jaron Robertson and Shaina Quinn of UTA and Sahar Shirazi and Kenny Ferrel of WSP for oversight and input throughout the project. }
%DIFDELCMD < 

%DIFDELCMD < %%%
%DIF < %%%%%%%%%%%%%%%%%%%%%%%%%%%%%%%%%%%%%%%%%
\DIFdelend \conflictsofinterest{The authors declare no conflict of interest.}

%%%%%%%%%%%%%%%%%%%%%%%%%%%%%%%%%%%%%%%%%%
%% optional
\DIFdelbegin %DIFDELCMD < \abbreviations{The following abbreviations are used in this manuscript:\\
%DIFDELCMD < 

%DIFDELCMD < \noindent 
%DIFDELCMD < \begin{tabular}{@{}ll}
%DIFDELCMD < TNC & Transportation Network Company, e.g. Uber, Lyft\\
%DIFDELCMD < UTA & Utah Transit Authority\\
%DIFDELCMD < \end{tabular}}
%DIFDELCMD < %%%
\DIFdelend \DIFaddbegin \abbreviations{The following abbreviations are used in this manuscript:\\

\noindent
\begin{tabular}{@{}ll}
TNC & Transportation Network Company, e.g.~Uber, Lyft \\
UTA & Utah Transit Authority \\
\end{tabular}}
\DIFaddend 


%%%%%%%%%%%%%%%%%%%%%%%%%%%%%%%%%%%%%%%%%%
% Citations and References in Supplementary files are permitted provided that they also appear in the reference list here.

%=====================================
% References, variant A: internal bibliography
%=====================================
%DIF > \reftitle{References}
%DIF > \begin{thebibliography}{999}
%DIF >  Reference 1
%DIF > \bibitem[Author1(year)]{ref-journal}
%DIF > Author1, T. The title of the cited article. {\em Journal Abbreviation} {\bf 2008}, {\em 10}, 142--149.
%DIF >  Reference 2
%DIF > \bibitem[Author2(year)]{ref-book}
%DIF > Author2, L. The title of the cited contribution. In {\em The Book Title}; Editor1, F., Editor2, A., Eds.; Publishing House: City, Country, 2007; pp. 32--58.
%DIF > \end{thebibliography}

% The following MDPI journals use author-date citation: Arts, Econometrics, Economies, Genealogy, Humanities, IJFS, JRFM, Laws, Religions, Risks, Social Sciences. For those journals, please follow the formatting guidelines on http://www.mdpi.com/authors/references
% To cite two works by the same author: \citeauthor{ref-journal-1a} (\citeyear{ref-journal-1a}, \citeyear{ref-journal-1b}). This produces: Whittaker (1967, 1975)
% To cite two works by the same author with specific pages: \citeauthor{ref-journal-3a} (\citeyear{ref-journal-3a}, p. 328; \citeyear{ref-journal-3b}, p.475). This produces: Wong (1999, p. 328; 2000, p. 475)

%=====================================
% References, variant B: external bibliography
%=====================================
\DIFaddbegin \reftitle{References}
\DIFaddend \externalbibliography{yes}
\DIFdelbegin %DIFDELCMD < \bibliography{citations}
%DIFDELCMD < %%%
\DIFdelend \DIFaddbegin \bibliography{book.bib}
\DIFaddend 

%%%%%%%%%%%%%%%%%%%%%%%%%%%%%%%%%%%%%%%%%%
%DIF > % optional
\DIFaddbegin 

%DIF > % for journal Sci
%DIF > \reviewreports{\\
%DIF > Reviewer 1 comments and authors’ response\\
%DIF > Reviewer 2 comments and authors’ response\\
%DIF > Reviewer 3 comments and authors’ response
%DIF > }

\DIFaddend %%%%%%%%%%%%%%%%%%%%%%%%%%%%%%%%%%%%%%%%%%
\end{document}

